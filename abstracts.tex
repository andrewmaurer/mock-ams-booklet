
\documentclass[oneside]{amsart}

\usepackage{amsthm,amssymb,amsmath}
\usepackage[margin=1in]{geometry}

%\usepackage{draftwatermark}
%\SetWatermarkScale{4}

\begin{document}

\title[Mock AMS]{Mock AMS Conference \\ Abstracts}
\date{July 26-27, 2018}
\author{University of Georgia \\ Department of Mathematics}
\maketitle
\noindent\rule{\textwidth}{0.4pt}
\vspace{0.5em}

\filbreak
\hspace{-20pt}\textbf{ \textbf{ Collatz Conjecture } } \vspace{0.5em}\\
This talk will introduce the Collatz Conjecture and some reformulations in an attempt to prove it. The main focus would be the analogue of the conjecture in polynomial rings of characteristic $2$. \vspace{-1em}\\
\begin{flushright} \textit{ Komal Agrawal } \vspace{0.5em} \end{flushright}
\rule{\textwidth}{0.4pt}
\vspace{0.5em}

\filbreak
\hspace{-20pt}\textbf{ \textbf{ A Mathematical Introduction to Quantum Mechanics } } \vspace{0.5em}\\
In this talk I will introduce the quantum mechanical position and momentum operators and discuss their mathematical subtleties. I will also discuss their commutation relation and it's implications. \vspace{-1em}\\
\begin{flushright} \textit{ Kenneth Allen } \vspace{0.5em} \end{flushright}
\rule{\textwidth}{0.4pt}
\vspace{0.5em}

\filbreak
\hspace{-20pt}\textbf{ \textbf{ About finite groups acting freely on $S^{2k-1}$ } } \vspace{0.5em}\\
We all know that $\mathbb{Z}_2$ is the only non-trivial group that can act freely on even dimensional spheres. We also know that there are other groups that can act freely on an odd dimensional sphere, for example it is easy to construct a free action of any cyclic group $\mathbb{Z}_m$ on $S^{2k-1}$, which gives rise to the lens spaces as the orbit space $S^{2k-1}/\mathbb{Z}_m$. But are there any other group that acts freely on an odd dimensional sphere? It turns out that there are other finite groups acting freely on $S^3$. In 1957, Milnor gave a characterization of finite groups acting freely on a sphere. In this talk we're going to discuss the proof of this characterization. \vspace{-1em}\\
\begin{flushright} \textit{ Swapnanil Banerjee } \vspace{0.5em} \end{flushright}
\rule{\textwidth}{0.4pt}
\vspace{0.5em}

\filbreak
\hspace{-20pt}\textbf{ \textbf{ Small prime power residues } } \vspace{0.5em}\\
Let $p$ be a prime number. For each positive integer $k$, it is widely believed that the smallest prime that is a $k$th power residue modulo $p$ should be $O(p^{\epsilon})$, for any $\epsilon>0$. Elliott has proved that such a prime is at most $p^{\frac{k-1}{4}+\epsilon}$, for each $\epsilon>0$. In this talk we will see how extended reciprocity laws can lead to a result on number of prime $k$th power residues which are less than the bound proved by Elliott. \vspace{-1em}\\
\begin{flushright} \textit{ Kubra Benli } \vspace{0.5em} \end{flushright}
\rule{\textwidth}{0.4pt}
\vspace{0.5em}

\filbreak
\hspace{-20pt}\textbf{ \textbf{ Differential Graded Algebras of Legendrian Knots } } \vspace{0.5em}\\
A Legendrian knot is a knot with a bit of extra structure. In 1997 Chekanov constructed an invariant of Legendrian knots using differential graded algebras (DGAs). In this talk we will explain how to associate a DGA to a Legendrian knot and why the resulting invariant is notable. Expect lots of pictures. \vspace{-1em}\\
\begin{flushright} \textit{ Sarah Blackwell } \vspace{0.5em} \end{flushright}
\rule{\textwidth}{0.4pt}
\vspace{0.5em}

\filbreak
\hspace{-20pt}\textbf{ \textbf{ Introduction to Grassmannians } } \vspace{0.5em}\\
A brief introduction to the Grassmannian and some of its presentations in topology, geometry, and algebraic geometry. \vspace{-1em}\\
\begin{flushright} \textit{ Joseph Dorta } \vspace{0.5em} \end{flushright}
\rule{\textwidth}{0.4pt}
\vspace{0.5em}

\filbreak
\hspace{-20pt}\textbf{ \textbf{ Galois theory and Covering Spaces } } \vspace{0.5em}\\
In the theory of covering spaces there exists what is called \textit{Galois correspondence} between connected covers of a topological space, and subgroups of the fundamental groups. More specifically, there is a bijective correspondence between the two that reverses inclusion. Conversely, in Galois Theory proper, there is a similar correspondence between intermediate fields of a Galois extension, and subgroups of the Galois group of automorphisms. The similarity between these two situations is not merely suggestive, indeed, the topic of this talk will be that there is a surprising connection between these two ideas that is revealed in the study of algebraic curves. \vspace{-1em}\\
\begin{flushright} \textit{ Amelia Ernst } \vspace{0.5em} \end{flushright}
\rule{\textwidth}{0.4pt}
\vspace{0.5em}

\filbreak
\hspace{-20pt}\textbf{ \textbf{ Ultrafilters and Arrow's Impossibility Theorem } } \vspace{0.5em}\\
Given a set $S$, a filter on $S$ is a subset of $\mathcal{P}(S)$ that contains $S$, does not contain $\emptyset$, and is closed under supersets and finite intersection.  An ultrafilter is a filter $U$  such that for all $A \in \mathcal{P}(S)$, either $A \in U$ or $A^{c} \in U$.

Ultrafilters have a wide variety of model-theoretic applications, including their use in ultraproduct constructions and in rigorously defining infinitesimals in non-standard analysis.  Intriguingly, they may also be applied in the proof of a result from social choice theory, Arrow's impossibility theorem, which states that no voting system can simultaneously satisfy unanimity, independence, and non-dictator conditions simultaneously for all elections with three or more candidates.  In this talk, we will define these terms formally and give a short proof of this result. \vspace{-1em}\\
\begin{flushright} \textit{ David Galban } \vspace{0.5em} \end{flushright}
\rule{\textwidth}{0.4pt}
\vspace{0.5em}

\filbreak
\hspace{-20pt}\textbf{ \textbf{ Computing the Fundamental Group of the Circle Using Van Kampen's Theorem } } \vspace{0.5em}\\
In this talk we will compute the fundamental group of the circle using Van Kampen's theorem. \vspace{-1em}\\
\begin{flushright} \textit{ Ernest Guico } \vspace{0.5em} \end{flushright}
\rule{\textwidth}{0.4pt}
\vspace{0.5em}

\filbreak
\hspace{-20pt}\textbf{ \textbf{ Synchrony and bistability in a Boolean network model of the L-arabinose operon } } \vspace{0.5em}\\
Mathematical models of gene regulatory networks have frequently been given in terms of systems of differential equations, but more recently discrete models such as Boolean networks have been used to model operons (segments of DNA containing several co-transcribed genes).  I will propose a Boolean network model for the L-arabinose operon, an operon that contains genes encoding proteins that catabolize the sugar arabinose.  By treating the network model as a polynomial dynamical system, we can perform an analysis of the system dynamics using computational algebra.  I will show that the model accurately captures the biological behavior of the operon and, in particular, the model appropriately exhibits fixed points or bistability based on the initial conditions.  Finally, I will mention differences that arise in the state space when using synchronous versus asynchronous update schemes. \vspace{-1em}\\
\begin{flushright} \textit{ Andy Jenkins } \vspace{0.5em} \end{flushright}
\rule{\textwidth}{0.4pt}
\vspace{0.5em}

\filbreak
\hspace{-20pt}\textbf{ \textbf{ Ribbon genus, Virtual graphs, and Wirtinger deficiency } } \vspace{0.5em}\\
The group of a knotted surface in 4-space is the fundamental group of its complement. We will give an elementary, constructive proof that every surface knot group is the group of an orientable ribbon surface. This allows us to define the ribbon genus of a knotted surface as the minimum genus of any ribbon surface with the same group. It will be shown that this topological property of a surface knot is equivalent to a property of its group. We conclude with several families of surface knots which realize their ribbon genus and a few applications. \vspace{-1em}\\
\begin{flushright} \textit{ Jason Joseph } \vspace{0.5em} \end{flushright}
\rule{\textwidth}{0.4pt}
\vspace{0.5em}

\filbreak
\hspace{-20pt}\textbf{ \textbf{ Subsequence Pattern Packing in Permutations and Compositions } } \vspace{0.5em}\\
Patterns are all around us, from the structure of the atoms in compounds to the motion of the planets above. It is the consistencies in nature that make possible the insights into her inner workings. The study of patterns in words has birthed from the need to send, process, and store large amounts of information in a dawning digital age. Concepts such as pattern avoidance and pattern packing have been studied extensively in permutations, and more recently in more general words such as compositions and preferential arrangements. An overview is given of some foundational results of subsequence pattern packing for permutations and compositions. Optimal permutations and compositions are then analyzed for specific patterns to give a measure of the packing ability for two different types of words. \vspace{-1em}\\
\begin{flushright} \textit{ Matt Just } \vspace{0.5em} \end{flushright}
\rule{\textwidth}{0.4pt}
\vspace{0.5em}

\filbreak
\hspace{-20pt}\textbf{ \textbf{ Existence of positive solution for fractional $p$ laplacian equation } } \vspace{0.5em}\\
I am going to introduce the fractional Laplacian and how to find the classical solution of the special type of fractional laplacian equation \vspace{-1em}\\
\begin{flushright} \textit{ Jinsil Lee } \vspace{0.5em} \end{flushright}
\rule{\textwidth}{0.4pt}
\vspace{0.5em}

\filbreak
\hspace{-20pt}\textbf{ \textbf{ Irreducible quadratic polynomials and Euler's function } } \vspace{0.5em}\\
We investigate the range of Euler's $\phi$-function. In 1929, Pillai proved that almost all numbers lie outside the range of $\phi(n)$. For a given polynomial $P$, we may ask whether $P(n)$ is outside the range of $\phi(n)$ as well. We discuss this problem for irreducible quadratic polynomials. \vspace{-1em}\\
\begin{flushright} \textit{ Noah Lebowitz-Lockard } \vspace{0.5em} \end{flushright}
\rule{\textwidth}{0.4pt}
\vspace{0.5em}

\filbreak
\hspace{-20pt}\textbf{ \textbf{ Debiasing Word Embeddings } } \vspace{0.5em}\\
Many machine learning applications rely on \emph{word embeddings}, i.e., a function $f: \operatorname{OED} \to \mathbb{R}^N$ which captures the geometry of word meanings. When the unbiased algorithms are trained on biased data, we inadvertently create biased artificial intelligences. This is evidenced by Microsoft's "Tay" Twitter bot, which was taught to send racist tweets. Additionally, Google's Word2Vec system completed the analogy ``Man is to Computer Programmer as Woman is to $x$'' with ``$x =$ homemaker''. This talk will discuss techniques for removing bias while maintaining utility in word embeddings. \vspace{-1em}\\
\begin{flushright} \textit{ Andrew Maurer } \vspace{0.5em} \end{flushright}
\rule{\textwidth}{0.4pt}
\vspace{0.5em}

\filbreak
\hspace{-20pt}\textbf{ \textbf{ Thinking Inside the (Distinct or Identical) Boxes: A Friendly Introduction to Object Distributions and Stirling Numbers of the Second Kind } } \vspace{0.5em}\\
Mathematicians are often concerned with how to count the distributions of (distinct or identical) objects into (distinct or identical) "boxes." This presentation provides an elementary description of object distribution problems, Stirling Numbers of the first and second kinds, Bell numbers, integer partitions, and combinatorial applications of these concepts. \vspace{-1em}\\
\begin{flushright} \textit{ Catrina May } \vspace{0.5em} \end{flushright}
\rule{\textwidth}{0.4pt}
\vspace{0.5em}

\filbreak
\hspace{-20pt}\textbf{ \textbf{ A Bivariate Spline Analysis of the TEM mode of a Parallel Plate Waveguide } } \vspace{0.5em}\\
Beginning with the Maxwell equations, we offer a brief explanation for the analysis of an idealized parallel-plate waveguide, including TE, TM, and TEM modes, and cutoff frequencies. When a dielectric obstruction is present, a scattering phenomenon is observed, and the governing PDE with continuity conditions must be solved to understand the full picture of the resulting electromagnetic waves. 

We use bivariate splines of arbitrary degree to solve the problem for a variety of dielectric obstructions. Further, we make use of a modified spline smoothness condition to explicitly enforce the physical continuity condition. Analysis of this enforcement shows that it compares well with the standard variational approach. \vspace{-1em}\\
\begin{flushright} \textit{ Clay Mersmann } \vspace{0.5em} \end{flushright}
\rule{\textwidth}{0.4pt}
\vspace{0.5em}

\filbreak
\hspace{-20pt}\textbf{ \textbf{ Fractional Coloring } } \vspace{0.5em}\\
A graph is $k$-colorable if and only if we can assign the value $0$ or $1$ to each independent set so that the sum of the assigned values is at most $k$ and each vertex is contained in an independent set having value $1$. In this talk we discuss the fractional relaxation of $k$-coloring. We define a fractional coloring of a graph and the fractional chromatic number. We also define the Kneser graphs and discuss the role these graphs fulfill analogous to that of the complete graphs for the ordinary chromatic number. \vspace{-1em}\\
\begin{flushright} \textit{ Kirsten Morris } \vspace{0.5em} \end{flushright}
\rule{\textwidth}{0.4pt}
\vspace{0.5em}

\filbreak
\hspace{-20pt}\textbf{ \textbf{ Sudoku Symmetry Group } } \vspace{0.5em}\\
We will discuss some interesting factoids about sudoku puzzles and their symmetries, prove some things, AND answer a question I've been asking myself for years. \vspace{-1em}\\
\begin{flushright} \textit{ Alex Newman } \vspace{0.5em} \end{flushright}
\rule{\textwidth}{0.4pt}
\vspace{0.5em}

\filbreak
\hspace{-20pt}\textbf{ \textbf{ Four-manifold invariants from broken Lefschetz fibrations and trisections } } \vspace{0.5em}\\
The topology of four-dimensional manifolds can be studied effectively via maps to standard surfaces. We compare two quintessential constructions: broken Lefschetz fibrations and trisections. Time permitting, we'll explain how these structures lead to Floer-theoretic invariants of four-manifolds. \vspace{-1em}\\
\begin{flushright} \textit{ William Olsen } \vspace{0.5em} \end{flushright}
\rule{\textwidth}{0.4pt}
\vspace{0.5em}

\filbreak
\hspace{-20pt}\textbf{ \textbf{ The Affine Low-Rank Matrix Completion Algorithm } } \vspace{0.5em}\\
The Affine Low-Rank Matrix Completion Algorithm (ALRMC) is a numerical algorithm for the recovery of an affine low-rank matrix corrupted by noise, errors, and erasures, based on the proximal forward-backward splitting algorithm. This recovery problem arises when solving the rigid motion segmentation problem, i.e. the problem of identifying which feature tracks in a video sequence belong to a common rigid object. Under orthography, feature tracks that all belong to a particular rigid object lie in a 4-dimensional affine subspace. Hence the interest in an effective algorithm to recover a low-dimensional affine subspace from the data points. With this application in mind, we allow for three sources of corruption in the data common to video: noise (low magnitude), errors (high magnitude but sparse), and erasures (missing data or high magnitude corruptions with known locations). \vspace{-1em}\\
\begin{flushright} \textit{ Eric Perkerson } \vspace{0.5em} \end{flushright}
\rule{\textwidth}{0.4pt}
\vspace{0.5em}

\filbreak
\hspace{-20pt}\textbf{ \textbf{ An Axiomatic Approach to Homology Theory In Algebraic Topology } } \vspace{0.5em}\\
A constructive approach to defining the homology groups of a space (or the relative homology of a pair of spaces), seen for example in Hatcher's popular text, has the benefit of explicitly detailing the geometry at play. However, one must work through the machinery of a constructed homology theory for some time before it becomes clear what its important pieces and properties are. An axiomatic approach, on the other hand, places the vital features at center stage by making them the defining properties of a homology theory.  

In this talk, we will discuss the setting for and state the defining Eilenberg-Steenrod axioms of a homology theory. After noting some common examples coming from algebraic topology, we will talk about uniqueness of homology theories, discuss a generalization, and see that we get the Mayer-Vietoris sequence result for all so called ``extraordinary'' homology theories. \vspace{-1em}\\
\begin{flushright} \textit{ Freddy Saia } \vspace{0.5em} \end{flushright}
\rule{\textwidth}{0.4pt}
\vspace{0.5em}

\filbreak
\hspace{-20pt}\textbf{ \textbf{ Geometry and Galois Theory } } \vspace{0.5em}\\
It is far from obvious that the theory of analytic continuation should have any connection to Galois theory. The goal of this talk is to explain how the two theories are indeed related. Beginning from a (seemingly) basic problem in complex analysis, we will see how one can arrive at analogues of Galois theory for topological spaces and Riemann surfaces, and how these theories connect to the classical correspondence for fields. \vspace{-1em}\\
\begin{flushright} \textit{ Nolan Schock } \vspace{0.5em} \end{flushright}
\rule{\textwidth}{0.4pt}
\vspace{0.5em}

\filbreak
\hspace{-20pt}\textbf{ \textbf{ Totally geodesic subspaces of the space of polygons in \(\mathbb{R}^3\) } } \vspace{0.5em}\\
In 1995, Allen Knutson and Jean-Claude Hausmann proved a connection between the Grassmann manifold \(G_2(C^n)\) and \(n\)-edge polygons in \(\mathbb{R}^3\), leading to several results on the differential structure of polygon space and its applications. In 2009, Sebastian Klein determined a complete list of totally geodesic subspaces of the Grassmann manifold. In my talk we will relate some of these totally geodesic subspaces to subspaces of the space of polygons in \(\mathbb{R}^3\), including lined polygons, planar polygons and recovering the result, that parallelograms in \(\mathbb{R}^2\) are volume minimizing and totally geodesic inside the Grassmannian of planar quadrilaterals. \vspace{-1em}\\
\begin{flushright} \textit{ Erik Schreyer } \vspace{0.5em} \end{flushright}
\rule{\textwidth}{0.4pt}
\vspace{0.5em}

\filbreak
\hspace{-20pt}\textbf{ \textbf{ Iwasawa Theory of Elliptic Curves } } \vspace{0.5em}\\
Iwasawa theory studies arithmetic objects through infinite towers of number fields. In this talk, I will outline how Iwasawa Theory of elliptic curves was used to prove special cases of Birch and Swinnerton-Dyer conjecture. \vspace{-1em}\\
\begin{flushright} \textit{ Makoto Suwama } \vspace{0.5em} \end{flushright}
\rule{\textwidth}{0.4pt}
\vspace{0.5em}

\filbreak
\hspace{-20pt}\textbf{ \textbf{ Hodge Theory, Period Maps and Period Domains } } \vspace{0.5em}\\
I will begin by motivating and defining Hodge structures in order to state the Hodge decomposition theorem for K\"{a}hler manifolds, with a few consequences to illustrate its power in algebraic/differential geometry.  I will then define the period map and how it relates to the parameter space of polarized Hodge structures (period domain) and polarized variations of Hodge structures. \vspace{-1em}\\
\begin{flushright} \textit{ Benjamin Tighe } \vspace{0.5em} \end{flushright}
\rule{\textwidth}{0.4pt}
\vspace{0.5em}

\filbreak
\hspace{-20pt}\textbf{ \textbf{ Tannakian Reconstruction of Finite Groups } } \vspace{0.5em}\\
Given all of the finite dimensional complex representations of a group, can we reconstruct the group? This talk will show how to do so for finite groups, and mention when it can be done for infinite groups. \vspace{-1em}\\
\begin{flushright} \textit{ Jack Wagner } \vspace{0.5em} \end{flushright}
\rule{\textwidth}{0.4pt}
\vspace{0.5em}

\filbreak
\hspace{-20pt}\textbf{ \textbf{ What makes a metric super? } } \vspace{0.5em}\\
We will examine some of the strange things that occur geometrically and analytically with the super metric. \vspace{-1em}\\
\begin{flushright} \textit{ Terrin Warren } \vspace{0.5em} \end{flushright}
\rule{\textwidth}{0.4pt}
\vspace{0.5em}

\filbreak
\hspace{-20pt}\textbf{ \textbf{ Hasse Principle Violations of Quadratic Twists of Hyperelliptic Curves } } \vspace{0.5em}\\
A curve $C/\mathbb{Q}$ is said to violate the Hasse Principle if $C$ has points over every completion of  $\mathbb{Q}$ but not over $\mathbb{Q}$  itself. Conditionally on the ABC conjecture, we show that if a hyperelliptic curve $C/\mathbb{Q}$ is given by $y^2 = f(x)$, where $f$ is a polynomial of even degree $> 6$ with integer coefficients and no rational roots, then there are many quadratic twists of $C$ violating the Hasse Principle. This is joint work with Pete L. Clark. \vspace{-1em}\\
\begin{flushright} \textit{ Lori Watson } \vspace{0.5em} \end{flushright}
\rule{\textwidth}{0.4pt}
\vspace{0.5em}

\filbreak
\hspace{-20pt}\textbf{ \textbf{ First Digits of $2^n$ } } \vspace{0.5em}\\
You might know a lot about the last digits of $2^n$ - but what about the first digits? The (asymptotic) distribution can - perhaps surprisingly - be explicitly described using introductory ergodic theory. I will give an overview of the result, the theorems behind it, and leave you with a reminder of how not to cheat on your taxes. \vspace{-1em}\\
\begin{flushright} \textit{ Peter Woolfitt } \vspace{0.5em} \end{flushright}
\rule{\textwidth}{0.4pt}
\vspace{0.5em}

\filbreak
\hspace{-20pt}\textbf{ \textbf{ An Overview of Compactifying Moduli Spaces } } \vspace{0.5em}\\
There are many different constructions of compactification of moduli spaces, I will talk about VGIT, KSBA, AMRT and GGLR. \vspace{-1em}\\
\begin{flushright} \textit{ Xian Wu } \vspace{0.5em} \end{flushright}
\rule{\textwidth}{0.4pt}
\vspace{0.5em}

\filbreak
\hspace{-20pt}\textbf{ \textbf{ Good subsets of the unit circle } } \vspace{0.5em}\\
Spherical designs are certain good subsets of the sphere that approximate the sphere. Many examples of spherical designs come from good geometric objects, say, regular polytopes, orbits and lattices. In this talk, I will focus on the spherical designs on the unit circle, and discuss some concrete open problems. \vspace{-1em}\\
\begin{flushright} \textit{ Ziqing Xiang } \vspace{0.5em} \end{flushright}
\rule{\textwidth}{0.4pt}
\vspace{0.5em}

\filbreak
\hspace{-20pt}\textbf{ \textbf{ Duality in Optimization } } \vspace{0.5em}\\
There are many algorithms in optimization, and duality is one of them. The dual problems can be motivated as a way to find bounds on a given optimization problem. Under rather mild conditions in the convex case, the strong duality holds. In this talk, we will go through some basic definitions and theorems in duality and do several examples to see how duality can be used to simplify or even solve problems. \vspace{-1em}\\
\begin{flushright} \textit{ Yidong Xu } \vspace{0.5em} \end{flushright}
\rule{\textwidth}{0.4pt}
\vspace{0.5em}


\end{document}

